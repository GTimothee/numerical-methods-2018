\documentclass[12pt]{article}

\usepackage[margin=1in]{geometry}
\usepackage{amsmath,amsthm,amssymb}
\usepackage{mathtools}
\usepackage{multicol}
\usepackage{textcomp}
\usepackage{float}
\usepackage{longtable}

\newcommand{\N}{\mathbb{N}}
\newcommand{\Z}{\mathbb{Z}}
\newcommand\aug{\fboxsep=-\fboxrule\!\!\!\fbox{\strut}\!\!\!}

\newenvironment{theorem}[2][Theorem]{\begin{trivlist}
\item[\hskip \labelsep {\bfseries #1}\hskip \labelsep {\bfseries #2.}]}{\end{trivlist}}
\newenvironment{lemma}[2][Lemma]{\begin{trivlist}
\item[\hskip \labelsep {\bfseries #1}\hskip \labelsep {\bfseries #2.}]}{\end{trivlist}}
\newenvironment{exercise}[2][Exercise]{\begin{trivlist}
\item[\hskip \labelsep {\bfseries #1}\hskip \labelsep {\bfseries #2.}]}{\end{trivlist}}
\newenvironment{reflection}[2][Reflection]{\begin{trivlist}
\item[\hskip \labelsep {\bfseries #1}\hskip \labelsep {\bfseries #2.}]}{\end{trivlist}}
\newenvironment{proposition}[2][Proposition]{\begin{trivlist}
\item[\hskip \labelsep {\bfseries #1}\hskip \labelsep {\bfseries #2.}]}{\end{trivlist}}
\newenvironment{corollary}[2][Corollary]{\begin{trivlist}
\item[\hskip \labelsep {\bfseries #1}\hskip \labelsep {\bfseries #2.}]}{\end{trivlist}}

\begin{document}

\title{TUTORIAL 7}
\author{Timothée Guédon \& Tristan Glatard\\
COMP 361 Numerical Methods}
\date{November 1st, 2019}
\maketitle

\section{Exercises for today}

\begin{exercise}{1}
  Use finite difference approximations of \(O(h^2)\) to compute \(f^\prime (2.36)\) and \(f^{\prime \prime} (2.36)\) from the following data:

  \begin{table}[h]
    \centering
    \begin{tabular}{|c|c|c|c|c|}
      \hline
      x & 2.36 & 2.37 & 2.38 & 2.39 \\ \hline
      f(x) & 0.85866 & 0.86289 & 0.86710 & 0.87129 \\ \hline
    \end{tabular}
  \end{table}
\end{exercise}

\begin{exercise}{2}
  Estimate \(f^\prime (1)\) and \(f^{\prime \prime} (1)\) from the following data:

  \begin{table}[h]
    \centering
    \begin{tabular}{|c|c|c|c|}
      \hline
      x & 0& 1& 3\\ \hline
      f(x) & -3 &0 &12 \\ \hline
    \end{tabular}
  \end{table}
\end{exercise}

\begin{exercise}{3}
  Calculate $f^{\prime \prime}(1)$ as accurately as you can given the following data
  \begin{table}[h]
    \centering
    \begin{tabular}{|c|c|c|c|c|c|}
      \hline
      x &0.84& 0.92& 1.00& 1.08& 1.16\\ \hline
      f (x) &0.431711 &0.398519& 0.367879& 0.339596 & 0.313486 \\ \hline
    \end{tabular}
  \end{table}
\end{exercise}

  \begin{exercise}{4}
  Use the data in the table to compute $f^\prime (0.2)$ as accurately as possible:
  \begin{table}[h]
    \centering
    \begin{tabular}{|c|c|c|c|c|c|}
      \hline
      x &0& 0.1& 0.2& 0.3& 0.4\\ \hline
      f(x)& 0.000 000& 0.078 348 &0.138 910& 0.192 916 &0.244 981 \\ \hline
    \end{tabular}
  \end{table}
\end{exercise}

\break

\section{Solutions}

%-------------------------------------------------------------------------------------------------------
\subsection{Exercise 1}
%-------------------------------------------------------------------------------------------------------

From the forward difference tables we get
\begin{align}
f^\prime(2.36) &\approx \frac{-3f(2.36) + 4f(2.37) - f(2.38)}{2(0.01)} \notag \\
&= \frac{-3*0.85866 + 4*0.86289-0.86710}{0.02} = 0.424 \notag \\
f^ {\prime \prime}(2.36) &\approx \frac{2f(2.36) -5f(2.37) + 4f(2.38) - f(2.39)}{0.01^2} \notag \\
&= \frac{2*0.85866 -5*0.86289+4*0.86710-0.87129}{0.0001} = -0.2 \notag
\end{align}

%-------------------------------------------------------------------------------------------------------
\subsection{Exercise 2}
%-------------------------------------------------------------------------------------------------------

Here we see the \textit{x} points are not evenly spaced, difference approximations cannot be applied.
Let's do it with polynomial interpolation, i.e. interpolate a polynomial of degree 2 (\textit{why 2?}) $P_2(x)$ and then approximate  \(f^\prime (1)\) and \(f^{\prime \prime} (1)\) by \(P_2^\prime (1)\) and \(P_2^{\prime \prime} (1)\), respectively.

You can use any interpolation method, e.g. Lagrange's, Newton, linear equations ..., to interpolate $P_2(x)$.

Review Tutorial 2 and apply \textit{Newton} we should get
\begin{align}
P_2(x)= x^2 + 2x -3 \notag
\end{align}

Then

\begin{align}
P^\prime_2(x)&= 2x+2 \notag \\
P^{\prime\prime}_2(x)&= 2 \notag
\end{align}


Finally
\begin{align}
f^\prime(1) &\approx P^\prime_2(1)= 2*1+2 = 4 \notag \\
f^{\prime\prime}(1)& \approx P^{\prime\prime}_2(1)= 2 \notag
\end{align}

%-------------------------------------------------------------------------------------------------------
\subsection{Exercise 3}
%-------------------------------------------------------------------------------------------------------

We start with two central difference approximations $O(h^2)$ for $f^{\prime \prime}(1)$, one using $h_1=0.16$ and the one using $h_2=0.08$.
\begin{align}
g(0.16)&=\frac{f(0.84) - 2f(1)+f(1.16)}{0.16^2} \notag \\
			&=\frac{0.431711 - 2*0.367879+0.313486}{0.16^2} \notag \\
            &=0.368710937 \notag \\
g(0.08)&=\frac{f(0.92) - 2f(1)+f(1.08)}{0.08^2} \notag \\
			&=\frac{0.398519 - 2*0.367879+0.339596}{0.08^2} \notag \\
            &=0.36828125 \notag
\end{align}

Recall that the error in both approximations is of the form \(E(h) = c_1h^2 + c_2h^4+c_3h_6 +...\) We can now use Richardson extrapolation to eliminate the dominant error term. With $p = 2$ we obtain
\begin{align}
f^{\prime \prime}(1) \approx G = \frac{2^2g(0.08)-g(0.16)}{2^2-1} = \frac{4*0.36828125-0.368710937}{3}
= 0.368138021 \notag
\end{align}
which is a finite difference approximation of \(O(h^4)\).

%-------------------------------------------------------------------------------------------------------
\subsection{Exercise 4}
%-------------------------------------------------------------------------------------------------------

We start with two central difference approximations $O(h^2)$ for $f^{\prime}(0.2)$, one using $h_1=0.2$ and the one using $h_2=0.1$.
\begin{align}
g(0.2)&=\frac{f(0.4) - f(0)}{2*0.2} \notag \\
			&=\frac{0.244981 - 0}{0.4} \notag \\
            &=0.6124525 \notag \\
g(0.1)&=\frac{f(0.3) - f(0.1)}{2*0.1} \notag \\
			&=\frac{0.192916 - 0.078348}{0.2} \notag \\
            &=0.57284 \notag
\end{align}

Recall that the error in both approximations is of the form \(E(h) = c_1h^2 + c_2h^4+c_3h_6 +...\) We can now use Richardson extrapolation to eliminate the dominant error term. With $p = 2$ we obtain
\begin{align}
f^{\prime}(0.2) \approx G = \frac{2^2g(0.1)-g(0.2)}{2^2-1} = \frac{4*(0.57284)-0.6124525}{3}
= 0.559635833 \notag
\end{align}
which is a finite difference approximation of \(O(h^4)\).

\end{document}
