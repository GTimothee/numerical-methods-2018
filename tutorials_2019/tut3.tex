\documentclass[12pt]{article}

\usepackage[margin=1in]{geometry}
\usepackage{amsmath,amsthm,amssymb}
\usepackage{mathtools}
\usepackage{multicol}
\usepackage{textcomp}
\usepackage{float}
\usepackage{longtable}

\newcommand{\N}{\mathbb{N}}
\newcommand{\Z}{\mathbb{Z}}
\newcommand\aug{\fboxsep=-\fboxrule\!\!\!\fbox{\strut}\!\!\!}

\newenvironment{theorem}[2][Theorem]{\begin{trivlist}
\item[\hskip \labelsep {\bfseries #1}\hskip \labelsep {\bfseries #2.}]}{\end{trivlist}}
\newenvironment{lemma}[2][Lemma]{\begin{trivlist}
\item[\hskip \labelsep {\bfseries #1}\hskip \labelsep {\bfseries #2.}]}{\end{trivlist}}
\newenvironment{exercise}[2][Exercise]{\begin{trivlist}
\item[\hskip \labelsep {\bfseries #1}\hskip \labelsep {\bfseries #2.}]}{\end{trivlist}}
\newenvironment{reflection}[2][Reflection]{\begin{trivlist}
\item[\hskip \labelsep {\bfseries #1}\hskip \labelsep {\bfseries #2.}]}{\end{trivlist}}
\newenvironment{proposition}[2][Proposition]{\begin{trivlist}
\item[\hskip \labelsep {\bfseries #1}\hskip \labelsep {\bfseries #2.}]}{\end{trivlist}}
\newenvironment{corollary}[2][Corollary]{\begin{trivlist}
\item[\hskip \labelsep {\bfseries #1}\hskip \labelsep {\bfseries #2.}]}{\end{trivlist}}

\begin{document}

\title{TUTORIAL 3}
\author{Timothée Guédon \& Tristan Glatard\\
COMP 361 Numerical Methods}
\date{September 27, 2019}
\maketitle

\section{Exercises for today}

\begin{exercise}{1}

  Determine y at x = 0 using Lagrange\textquotesingle s method for the given data points: \\
  \begin{table}[h]
  \centering
  \begin{tabular}{|c|c|c|c|}
    \hline
    x & -1.2 & 0.3 & 1.1 \\ \hline
    y & -5.76 & -5.61 & -3.69 \\ \hline
  \end{tabular}
  \end{table}

\end{exercise}

\begin{exercise}{2}

  Find the Doolittle's LU decomposition of the following matrix: \\

  $$A=\begin{bmatrix}
    2 & 1 & 0 & 0 & 0 \\
    -1 & 2 & 2 & 0 & 0 \\
    0 & -1 & 2 & 3 & 0 \\
    0 & 0 & -1 & 2 & 4 \\
    0 & 0 & 0 & -1 & 2
  \end{bmatrix}$$

\end{exercise}

\begin{exercise}{3} %You can use theorem, proposition, exercise, or reflection here.
  Use Newton\textquotesingle s method to find the polynomial that fits the following points:\\
  \begin{table}[h]
    \centering
    \begin{tabular}{|c|c|c|c|c|c|}
      \hline
      x & -5 & -1 & 0 & 2 \\ \hline
      y & -2 & 6 & 1 & 3 \\ \hline
    \end{tabular}
  \end{table}
\end{exercise}

\begin{exercise}{4} %You can use theorem, proposition, exercise, or reflection here.
  Use Newton\textquotesingle s method to find the polynomial that fits the following points:\\
  \begin{table}[h]
    \centering
    \begin{tabular}{|c|c|c|c|c|c|}
      \hline
      x & -3 & 2 & -1 & 3 & 1 \\ \hline
      y & 0 & 5 & -4 & 12 & 0 \\ \hline
    \end{tabular}
  \end{table}
\end{exercise}

\break

\section{Solutions}

%-------------------------------------------------------------------------------------------------------
\subsection{Exercise 1}
%-------------------------------------------------------------------------------------------------------

With 3 given data points, we can construct a degree-2 polynomial using the Lagrange formula:\\
$$P_{2}(x)=\sum_{i=0}^2 y_{i}\ell_{i}(x)$$
where
\begin{align}
\ell_{0}(x)&=\frac{(x-x_1)(x-x_2)}{(x_0-x_1)(x_0-x_2)}=\frac{(x-0.3)(x-1.1)}{3.45} \\
\ell_{1}(x)&=\frac{(x-x_0)(x-x_2)}{(x_1-x_0)(x_1-x_2)}=\frac{(x+1.2)(x-1.1)}{-1.2} \\
\ell_{2}(x)&=\frac{(x-x_0)(x-x_1)}{(x_2-x_0)(x_2-x_1)}=\frac{(x+1.2)(x-0.3)}{1.84}
\end{align} \\
Thus,
\begin{align}
\notag
P_{n}(x) &=y_0\ell_0(x) + y_1\ell_1(x) + y_2\ell_2(x) \\ \notag
P_{n}(0) &=y_0\ell_0(0) + y_1\ell_1(0) + y_2\ell_2(0) \\ \notag
P_{n}(0) &=-5.76 \frac{(-0.3)(-1.1)}{3.45} -5.61 \frac{(1.2)(-1.1)}{-1.2} - 3.69 \frac{(1.2)(-0.3)}{1.84} \\ \notag
P_{n}(0) &= -0.55-6.171+782 = -5.939 \notag
\end{align}

%-------------------------------------------------------------------------------------------------------
\subsection{Exercise 2}
%-------------------------------------------------------------------------------------------------------

The matrix is in the form:
$$\begin{bmatrix}
  d_1 & e_1 & 0 & 0 & 0 \\
  c_1 & d_2 & e_2 & 0 & 0 \\
  0 & c_2 & d_3 & e_3 & 0 \\
  0 & 0 & c_3 & d_4 & e_4 \\
  0 & 0 & 0 & c_4 & d_5
\end{bmatrix}$$ \\

\noindent Let's take the first step of Gauss elimination:

$$ c_1 - \lambda_1 d_1 = 0 $$
$$ \lambda = \frac{c_1}{d_1} $$
$$ L_2 \gets L_2 - \lambda_1 L_1 $$

$\begin{cases}
  c_1 \gets 0 \\
  d_2 \gets d_2 - \lambda_1 e_1 \\
  e_1 \gets e1  \\
\end{cases}$ \\

Like before, we can store the lambdas in $c_1$. It will be the same pattern to update each $d_{k+1}$, $e_k$ and $c_k$ for $k$ in $[1, n-1]$: \\

$\begin{cases}
  c_k \gets \lambda_k \\
  d_{k+1} \gets d_{k+1} - \lambda_k e_k \\
  e_k \gets e_k  \\
\end{cases}$ \\

We can therefore use a simple algorithm by storing the coefficients in three vectors instead of storing the whole matrix:

$$c = \begin{bmatrix}
  -1 \\
  -1 \\
  -1 \\
  -1
\end{bmatrix}, d = \begin{bmatrix}
  2 \\
  2 \\
  2 \\
  2 \\
  2
\end{bmatrix}, e = \begin{bmatrix}
  1 \\
  2 \\
  3 \\
  4
\end{bmatrix}$$ \\

\textbf{Step 1} ($k=1$): \\
$ \lambda_1 = \frac{c_1}{d_1} = \frac{-1}{2} $ \\
$ c_1 \gets \frac{-1}{2} $ \\
$ d_2 \gets 2 - \frac{-1}{2} 1 = \frac{5}{2} $ \\


$$c = \begin{bmatrix}
  \frac{-1}{2} \\
  -1 \\
  -1 \\
  -1
\end{bmatrix}, d = \begin{bmatrix}
  2 \\
  \frac{5}{2} \\
  2 \\
  2 \\
  2
\end{bmatrix}$$

\textbf{Step 2} ($k=2$): \\
$ \lambda_2 = \frac{c_2}{d_2} = \frac{-1}{\frac{5}{2}} = \frac{-2}{5} $ \\
$ c_2 \gets \frac{-2}{5} $ \\
$ d_3 \gets 2 - \frac{-2}{5} * 2 = \frac{14}{5} $ \\

$$c = \begin{bmatrix}
  \frac{-1}{2} \\
  \frac{-2}{5} \\
  -1 \\
  -1
\end{bmatrix}, d = \begin{bmatrix}
  2 \\
  \frac{5}{2} \\
  \frac{14}{5} \\
  2 \\
  2
\end{bmatrix}$$

\textbf{Step 3} ($k=3$): \\
$ \lambda_3 = \frac{c_3}{d_3} = \frac{-1}{\frac{14}{5}} = \frac{-5}{14} $ \\
$ c_3 \gets \frac{-5}{14} $ \\
$ d_4 \gets 2 - \frac{-5}{14} * 3 = \frac{43}{14} $ \\

$$c = \begin{bmatrix}
  \frac{-1}{2} \\
  \frac{-2}{5} \\
  \frac{-5}{14} \\
  -1
\end{bmatrix}, d = \begin{bmatrix}
  2 \\
  \frac{5}{2} \\
  \frac{14}{5} \\
  \frac{43}{14} \\
  2
\end{bmatrix}$$

\textbf{Step 4} ($k=4$): \\
$ \lambda_4 = \frac{c_4}{d_4} = \frac{-1}{\frac{43}{14}} = \frac{-14}{43} $ \\
$ c_4 \gets \frac{-14}{43} $ \\
$ d_5 \gets 2 - \frac{-14}{43} * 4 = \frac{86 + 56}{43} = \frac{142}{43} $ \\

$$c = \begin{bmatrix}
  \frac{-1}{2} \\
  \frac{-2}{5} \\
  \frac{-5}{14} \\
  \frac{-14}{43}
\end{bmatrix}, d = \begin{bmatrix}
  2 \\
  \frac{5}{2} \\
  \frac{14}{5} \\
  \frac{43}{14} \\
  \frac{142}{43}
\end{bmatrix}$$ \\

$$L = \begin{bmatrix}
  1 & 0 & 0 & 0 & 0 \\
  \frac{-1}{2} & 1 & 0 & 0 & 0 \\
  0 & \frac{-2}{5} & 1 & 0 & 0 \\
  0 & 0 & \frac{-5}{14} & 1 & 0 \\
  0 & 0 & 0 & \frac{-14}{43} & 1
\end{bmatrix}, U = \begin{bmatrix}
  2 & 1 & 0 & 0 & 0 \\
  0 & \frac{5}{2} & 2 & 0 & 0 \\
  0 & 0 & \frac{14}{5} & 3 & 0 \\
  0 & 0 & 0 & \frac{43}{14} & 4 \\
  0 & 0 & 0 & 0 & \frac{142}{43}
\end{bmatrix}$$ \\

%-------------------------------------------------------------------------------------------------------
\subsection{Exercise 3}
%-------------------------------------------------------------------------------------------------------

Construct the table of divided differences as follow: \\

\begin{table}[H]
\centering
\begin{tabular}{|c|c|c|c|c|c|c|}
\hline
\textit{i} & $x_i$ & $y_i$ & $\nabla y_i$ & $\nabla ^2 y_i$ & $\nabla ^3 y_i$ \\ \hline
0 & -5 & -2  &  &  & \\ \hline
1 & -1 & 6 & 2 &  &   \\ \hline
2 & 0 & 1 & $\frac{3}{5}$ & $\frac{-7}{5}$ &  \\ \hline
3 & 2 & 3 & $\frac{5}{7}$ & $\frac{-3}{7}$ & $\frac{17}{35}$  \\ \hline
\end{tabular}
\end{table}

where elements are calculated as:\\
\begin{align}
\notag
\nabla y_1 &= \frac{y_1-y_0}{x_1-x_0}=\frac{6+2}{-1+5}=2 \\
\notag
\nabla y_2 &= \frac{y_2-y_0}{x_2-x_0}=\frac{1+2}{0+5}=\frac{3}{5} \\
\notag
\nabla y_3 &= \frac{y_3-y_0}{x_3-x_0}=\frac{3+2}{2+5}=\frac{5}{7} \\
\notag
\nabla ^2 y_2 &= \frac{\nabla y_2-\nabla y_1}{x_2-x_1}=\frac{\frac{3}{5} - 2}{0+1}=\frac{\frac{3}{5} - \frac{10}{5}}{1}=\frac{-7}{5} \\
\notag
\nabla ^2 y_3 &= \frac{\nabla y_3-\nabla y_1}{x_3-x_1}=\frac{\frac{5}{7} - 2}{2+1}=\frac{-9}{21}=\frac{-3}{7} \\
\notag
\nabla ^3 y_3 &= \frac{\nabla ^2 y_3-\nabla ^2 y_2}{x_3-x_2}=\frac{\frac{-3}{7} + \frac{7}{5}}{2-0}=\frac{17}{35} \\
\notag
\end{align}

The formula used to do the calculations of the divided differences above is: \\

$ \nabla ^n y_i = \frac{\nabla ^{n-1} y_i - \nabla ^{n-1} y_{n-1}}{x_i-x_{n-1}} $ \\

For the first $n$ values we get: \\

$ \nabla ^1 y_i = \nabla y_i = \frac{\nabla ^0 y_i - \nabla ^0 y_{0}}{x_i-x_{0}} = \frac{y_i-y_0}{x_i-x_0} $ \\

$ \nabla ^2 y_i = \frac{\nabla y_i - \nabla y_1}{x_i-x_1} $ \\

$ \nabla ^3 y_i = \frac{\nabla ^2 y_i - \nabla ^2 y_2}{x_i-x_2} $ \\

Etc.

%-------------------------------------------------------------------------------------------------------
\subsection{Exercise 4}
%-------------------------------------------------------------------------------------------------------

Construct the table for Newton\textquotesingle s as follow

\begin{table}[H]
\centering
\begin{tabular}{|c|c|c|c|c|c|c|}
\hline
\textit{i} & $x_i$ & $y_i$ & $\nabla y_i$ & $\nabla ^2 y_i$ & $\nabla ^3 y_i$ & $\nabla ^4 y_i$ \\ \hline
0 & -3 & \textbf{0} &  &  &  &  \\ \hline
1 & 2 & 5 & \textbf{1} &  &  &  \\ \hline
2 & -1 & -4 & -2 & \textbf{1} &  &  \\ \hline
3 & 3 & 12 & 2 & 1 & \textbf{0} &  \\ \hline
4 & 1 & 0 & 0 & 1 & 0 & \textbf{0} \\ \hline
\end{tabular}
\end{table}

where elements are calculated as:\\
\begin{align}
\notag
\nabla y_1 &= \frac{y_1-y_0}{x_1-x_0}=\frac{5-0}{2+3}=1\\
\notag
\nabla y_2 &= \frac{y_2-y_0}{x_2-x_0}=\frac{-4+0}{-1+3}=-2\\
\notag
\nabla y_3 &= \frac{y_3-y_0}{x_3-x_0}=\frac{12-0}{3+3}=2\\
\notag
\nabla y_4 &= \frac{y_4-y_0}{x_4-x_0}=0\\
\notag
\nabla ^2 y_2 &= \frac{\nabla y_2-\nabla y_1}{x_2-x_1}=\frac{-2-1}{-1-2}=1\\
\notag
\nabla ^2 y_3 &= \frac{\nabla y_3-\nabla y_1}{x_3-x_1}=\frac{2-1}{3-2}=1\\
\notag
\nabla ^2 y_4 &= \frac{\nabla y_4-\nabla y_1}{x_4-x_1}=\frac{0-1}{1-2}=1\\
\notag
\nabla ^3 y_3 &= \frac{\nabla ^2 y_3-\nabla ^2 y_2}{x_3-x_2}=0\\
\notag
\nabla ^3 y_4 &= \frac{\nabla ^2 y_4-\nabla ^2 y_2}{x_4-x_2}=0\\
\notag
\nabla ^4 y_4 &= \frac{\nabla ^3 y_4-\nabla ^3 y_3}{x_4-x_3}=0\\
\notag
\end{align}
From the tableau \textquotesingle s diagonal, we have\\
\begin{align}
\notag
a_0&=y_0=0\\
\notag
a_1&=\nabla y_1=1\\
\notag
a_2&=\nabla ^2 y_2=1 \\
\notag
a_3&=\nabla ^3 y_3=0 \\
\notag
a_4&=\nabla ^4 y_4=0 \\
\end{align}
and the polynomial fit is degree-2 (parabolic).
Evaluate backward with recurrence relations:
\begin{align}
\notag
P_0(x) &= a_2 = 1\\
\notag
P_1(x) &= a_1 + (x-x_1)P_0(x) = 1 + (x-2) = x-1\\
\notag
P_2(x) &= a_0 + (x-x_0)P_1(x) = 0 + (x+3)(x-1) = x^2+2x-3\\
\notag
\end{align}

\end{document}
