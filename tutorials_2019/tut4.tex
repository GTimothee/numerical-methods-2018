\documentclass[12pt]{article}

\usepackage[margin=1in]{geometry}
\usepackage{amsmath,amsthm,amssymb}
\usepackage{mathtools}
\usepackage{multicol}
\usepackage{textcomp}
\usepackage{float}
\usepackage{longtable}

\newcommand{\N}{\mathbb{N}}
\newcommand{\Z}{\mathbb{Z}}
\newcommand\aug{\fboxsep=-\fboxrule\!\!\!\fbox{\strut}\!\!\!}

\newenvironment{theorem}[2][Theorem]{\begin{trivlist}
\item[\hskip \labelsep {\bfseries #1}\hskip \labelsep {\bfseries #2.}]}{\end{trivlist}}
\newenvironment{lemma}[2][Lemma]{\begin{trivlist}
\item[\hskip \labelsep {\bfseries #1}\hskip \labelsep {\bfseries #2.}]}{\end{trivlist}}
\newenvironment{exercise}[2][Exercise]{\begin{trivlist}
\item[\hskip \labelsep {\bfseries #1}\hskip \labelsep {\bfseries #2.}]}{\end{trivlist}}
\newenvironment{reflection}[2][Reflection]{\begin{trivlist}
\item[\hskip \labelsep {\bfseries #1}\hskip \labelsep {\bfseries #2.}]}{\end{trivlist}}
\newenvironment{proposition}[2][Proposition]{\begin{trivlist}
\item[\hskip \labelsep {\bfseries #1}\hskip \labelsep {\bfseries #2.}]}{\end{trivlist}}
\newenvironment{corollary}[2][Corollary]{\begin{trivlist}
\item[\hskip \labelsep {\bfseries #1}\hskip \labelsep {\bfseries #2.}]}{\end{trivlist}}

\begin{document}

\title{TUTORIAL 4}
\author{Timothée Guédon \& Tristan Glatard\\
COMP 361 Numerical Methods}
\date{October 4, 2019}
\maketitle

\section{Exercises for today}

\begin{exercise}{1}
Corrected exercise 3.7 from the textbook (p.123).
\end{exercise}

\begin{exercise}{2}
  Determine y at x = 1.5 using Lagrange\textquotesingle s method for the given data points: \\
  \begin{table}[h]
  \centering
  \begin{tabular}{|c|c|c|c|}
    \hline
    x & 0 & 3 & 6 \\ \hline
    y & 1.225 & 0.905 & 0.652 \\ \hline
  \end{tabular}
  \end{table}
\end{exercise}

\break

\section{Solutions}

%-------------------------------------------------------------------------------------------------------
\subsection{Exercise 1}
%-------------------------------------------------------------------------------------------------------

See textbook p.123

%-------------------------------------------------------------------------------------------------------
\subsection{Exercise 2}
%-------------------------------------------------------------------------------------------------------

With 3 given data points, we can construct a degree-2 polynomial using the Lagrange formula:\\
$$P_{2}(x)=\sum_{i=0}^2 y_{i}\ell_{i}(x)$$
where
\begin{align}
\ell_{0}(x)&=\frac{(x-x_1)(x-x_2)}{(x_0-x_1)(x_0-x_2)}=\frac{(x-3)(x-6)}{18} \\
\ell_{1}(x)&=\frac{(x-x_0)(x-x_2)}{(x_1-x_0)(x_1-x_2)}=\frac{(x-0)(x-6)}{-9} \\
\ell_{2}(x)&=\frac{(x-x_0)(x-x_1)}{(x_2-x_0)(x_2-x_1)}=\frac{(x-0)(x-3)}{18}
\end{align} \\
Thus,
\begin{align}
\notag
P_{2}(x) &=y_0\ell_0(x) + y_1\ell_1(x) + y_2\ell_2(x) \\ \notag
P_{2}(1.5) &=y_0\ell_0(1.5) + y_1\ell_1(1.5) + y_2\ell_2(1.5) \\ \notag
P_{2}(0) &= 1.225 \frac{(-1.5)(-4.5)}{18} + 0.925 \frac{(1.5)(-4.5)}{-9} + 0.652 \frac{(1.5)(-1.5)}{18} \\ \notag
P_{2}(0) &= 0.459375 + 0.67875 - 0.0815  = 1.056625 \notag
\end{align}

\end{document}
