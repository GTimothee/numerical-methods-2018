\documentclass{llncs}

\usepackage{amsmath} % for equation*
\usepackage{wasysym} % for \Box
\usepackage{tipa} % for |
\usepackage{color}
\usepackage{hyperref}
\usepackage{graphicx}
\usepackage{minted}
\definecolor{darkgreen}{rgb}{0,0.7,0}

% Fix link colors
\hypersetup{
    colorlinks = true,
    linkcolor=red,
    citecolor=red,
    urlcolor=blue,
    linktocpage % so that page numbers are clickable in toc
}

\pagestyle{plain}

\newcounter{ques}
\setcounter{ques}{1}

\newcommand{\todo}[1]{\color{blue}\textbf{TODO:} #1\color{black}}
\newcommand{\myspace}[0]{\vspace*{0.25cm}}

\renewcommand{\question}[1]{\paragraph{}\textbf{Q\theques} - #1\stepcounter{ques} }

\newcommand{\answer}[1]{\color{red}\textbf{Solution:}\\#1\color{black}}
\title{COMP 361/5611, Elementary Numerical Methods \\ Test 1 \\ Fall 2018 \\ Friday, October 19, 2018 \\ Duration: 150 minutes}

%\def\arraystretch{2}

\author{Tristan Glatard\\
  \href{mailto:tristan.glatard@concordia.ca}{tristan.glatard@concordia.ca}\\
  \vspace*{0.3cm}
  }

\institute{Concordia University\\
  Department of Computer Science and Software Engineering}

\begin{document}

\maketitle

\section*{Instructions}
\begin{itemize}
\item All questions will receive equal points.
\item Answer all questions on these sheets in the space provided.
\item No books, notes or extra paper (except draft paper received from the instructor).
\item No cell phones, laptops or any other electronic devices except ENCS calculators.
\item This exam is 13 pages long, including the cover page. It has
  12 questions labeled from \textbf{Q1} to \textbf{Q12}. Check that your copy
  is complete.
%~ \item Grading scheme:
%~ \begin{itemize}
%~ \item Good method, good result: 4pts
%~ \item Good method, wrong result (e.g., calculation error): 3pts
%~ \item Wrong method, good result: 2pts
%~ \item Wrong method, wrong result: 0pts
%~ \end{itemize}
\end{itemize}

\myspace

\myspace

\hrulefill\\

\myspace

Consistent with the university regulations concerning cheating and plagiarism I will not cheat during this examination:

\myspace

\myspace

Student ID: \dotfill

\myspace

\myspace

First Name / Last Name: \dotfill

\myspace

\myspace

Signature: \dotfill

\myspace

\myspace

\hrulefill

\newpage
%%% Chapter 2 (x2)
% 0. Conditioning
\question{By evaluating the determinant, determine if the following matrix is singular, ill-conditioned, or well-conditioned:
$$
\textbf{A}= 
\begin{bmatrix}
4 & 6 & -2 \\
1 & 9 & 4 \\
-1 & 1 & 7
\end{bmatrix}
$$
}
\answer{
$$
det\left( \textbf{A} \right) = 4 \left| 
\begin{tabular}{cc}
9 & 4 \\
1 & 7
\end{tabular}
 \right|
 - 6 \left| \begin{tabular}{cc}
 1 & 4 \\
 -1 & 7
 \end{tabular}
 \right|
 - 2 \left| \begin{tabular}{cc}
 1 & 9 \\
 -1 & 1
 \end{tabular}
 \right|
 = 4 \times (63 - 4) - 6 \times (7 + 4) -2 \times (1 + 9)
 = 4 \times 59 - 6 \times 11 - 2 \times 10 = 150
$$
The determinant is large compared to the matrix coefficients: the matrix
is well-conditioned.

Grading:
\begin{itemize}
\item 1pt for determinant calculation (partial mark if good method but calculation error, partial mark for sign error in determinant),
\item 1pt for conclusion on conditioning (no partial mark).
\end{itemize}

General comments on grading:
\begin{itemize}
\item Partial marks mean 0.5, we don't go to 0.25 in any question, unless otherwise mentioned.
\item Final mark /24 is converted /100 and rounded up to the nearest integer.
\end{itemize}
}



\newpage
% 1. Gauss elimination
\question{Use Gauss elimination to solve the equations $\textbf{Ax}=\textbf{b}$, where:
$$
\textbf{A}= 
\begin{bmatrix}
1 & 3 & 4 \\
3 & 6 & 14 \\
3 & -6 & 7
\end{bmatrix}
\quad
\textbf{b}= 
\begin{bmatrix}
3  \\
9  \\
0  \\
\end{bmatrix}
$$
}

\answer{
Elimination phase:

\noindent (2) $\leftarrow$ (2) - 3 $\times$ (1)\\
(3) $\leftarrow$ (3) - 3 $\times$ (1):\\
$$
\left(
\begin{array}{ccc|c}
1 & 3   & 4  & 3 \\
0 & -3  & 2  & 0 \\
0 & -15 & -5 & -9 
\end{array}
\right)
$$

\noindent (3) $\leftarrow$ (3) - 5 $\times$ (2):
$$
\left(
\begin{array}{ccc|c}
1 & 3   & 4  & 3 \\
0 & -3  & 2  & 0 \\
0 & 0 & -15 & -9 
\end{array}
\right)
$$

Backsubstitution gives:
$$
x_1 = -3/5 \quad ; \quad x_2 = 2/5 \quad ; \quad x_3 = 3/5
$$

Grading:
\begin{itemize}
\item 1pt for the method (no partial marks).
\item 1pt for the numerical result (partial marks if some iterations are correct).
\end{itemize}
}

\newpage
% 2. LU decomposition
\question{Use Doolittle's decomposition to find $\textbf{L}$ and $\textbf{U}$ such that:
$$
\textbf{A}=\textbf{LU}= 
\begin{bmatrix}
2 & 1 & 0 \\
3 & 8 & 1 \\
5 & 9 & 3
\end{bmatrix}
$$
}

\answer{
We perform Gauss elimination and store the multipliers ($\lambda$) in place
of the zeros of the matrix:

\noindent (2) $\leftarrow$ (2) - 3/2 $\times$ (1)\\
(3) $\leftarrow$ (3) - 5/2 $\times$ (1):
$$
\left[ \textbf{L} \backslash \textbf{U} \right] = 
\begin{bmatrix}
2 & 1 & 0 \\
\frac{3}{2} & \frac{13}{2} & 1 \\
\frac{5}{2} & \frac{13}{2} & 3
\end{bmatrix}
$$

\noindent (3) $\leftarrow$ (3) - (2):
$$
\left[ \textbf{L} \backslash \textbf{U} \right] = 
\begin{bmatrix}
2 & 1 & 0 \\
\frac{3}{2} & \frac{13}{2} & 1 \\
\frac{5}{2} & 1 & 2
\end{bmatrix}
$$

Finally:
$$
\textbf{L} =
\begin{bmatrix}
1 & 0 & 0 \\
\frac{3}{2} & 1 & 0 \\
\frac{5}{2} & 1 & 1
\end{bmatrix}
$$
and
$$
\textbf{U} = 
\begin{bmatrix}
2 & 1 & 0 \\
0 & \frac{13}{2} & 1 \\
0 & 0 & 2
\end{bmatrix}
$$

Grading:
\begin{itemize}
\item 1pt for the method (no partial mark).
\item 1pt for the numerical result (partial marks if some iterations are correct).
\end{itemize}
}


\newpage
% 3. Gauss elimination (with pivoting)
\question{Use the results of LU decomposition 
$$
\textbf{A}=\textbf{LU}= 
\begin{bmatrix}
1 & 0 & 0 \\
-3 & 1 & 0 \\
6 & 1 & 1
\end{bmatrix}
\begin{bmatrix}
-2 & 0 & 0 \\
0 & 1 & 1 \\
0 & 0 & 4
\end{bmatrix}
$$
to solve $\textbf{Ax}=\textbf{b}$ where $\textbf{b} = [1, 2, 3]^T$.
}

\answer{
We start by solving \textbf{Ly}=\textbf{b} by forward substitution:
$$
y_1 = 1
$$
$$
-3 y_1 + y_2 = 2 \Rightarrow y_2 = 5
$$
$$
6y_1 + y_2 + y_3 = 3 \Rightarrow y_3 = -8
$$

Then we solve \textbf{Ux}=\textbf{y} by back substitution:
$$
4x_3 = -8 \Rightarrow x_3 = -2
$$
$$
x_2 + x_3 = 5 \Rightarrow x_2 = 7
$$
$$
-2x_1 = 1 \Rightarrow x_1 = -1/2
$$


Grading:
\begin{itemize}
\item 1pt for the method (no partial mark).
\item 1pt for the numerical result (partial mark for correct y but incorrect x)
\end{itemize}
}


\newpage
%%% Chapter 3
% 4. Lagrange interpolation
\question{Given the data points:\\
\begin{tabular}{|c||c|c|c|}
\hline
x & 0 & 1 & 2 \\
\hline
y & 1 & 0 & 2\\
\hline
\end{tabular}\\
determine y at x=0.5 using Lagrange's method.
}

\answer{
We use Lagrange's formula, which we retrieve as follows: it's a linear combination
of polynomials $l_i$ where $l_i$ is 0 on all the $x_j$ for $j \neq i$ and 
1 at $x_i$. The linear combination is weighted by the $y_i$ to respect
the interpolation condition:

$$
P(x) = y_0 \frac{(x-x_1)(x-x_2)}{(x_0-x_1)(x_0-x_2)} +
       y_1 \frac{(x-x_0)(x-x_2)}{(x_1-x_0)(x_1-x_2)} +
       y_2 \frac{(x-x_0)(x-x_1)}{(x_2-x_0)(x_2-x_1)}
$$
It gives:
$$
P(x) = \frac{(x-1)(x-2)}{2} + x(x-1)
$$
We check that P respects the interpolation conditions: P(0)=1, P(1)=0 and P(2)=2. 
Since the interpolant is unique, we can't be wrong!

We then obtain:
$$
P(0.5) = \frac{-0.5*(-1.5)}{2} + 0.5*(-0.5) = 1/8
$$

Grading:
\begin{itemize}
\item 1pt for the method (with partial mark for a formula that ``looks good").
\item 1pt for the numerical result (no partial mark).
\end{itemize}
}

\newpage
% 5. Least-square regression
\question{Fill the missing part of the following Python function, between
the assignment of variable \texttt{x\_bar} and the \texttt{return} statement:}

\begin{minted}{python}
from numpy import array, mean, sum

def linear_regression(x_data, y_data):
    '''
    Returns a tuple (a, b) representing the straight line of equation
    y = a + bx that fits, in the least-squares sense, the points 
    represented by arrays x_data and y_data. 
    '''

    x_bar = mean(x_data)






















    return (a, b)

\end{minted}
\answer{See lecture notes. 

Grading:
\begin{itemize}
\item 2/3 pts for each missing line: y\_bar, b, a. Partial marks for approximate answers for b or a.
\end{itemize}
}


%%% Chapter 4 (x2)

\newpage
% 6. Bisection
\question{\underline{Using the bisection method}, find a zero of function $f(x)=\cos(x/2)$ assuming an initial bracket of [2.5, 3.5],
and using an accuracy of 0.1.}

\answer{
We start by noting that the zero of $f$ is $\pi \approx 3.14159$. This 
will help us check that our 
bracket is correct at each iteration (of course an algorithm couldn't use
this information, we will only use it to check our calculations!). We will define c=(a+b)/2 
throughout the solution.

\emph{Iteration 1:} a = 2.5, b=3.5, accuracy = 1\\
f(a) $>$ 0 ; f(b) $<$ 0.
f(c) = f(3) $>$ 0 $\Rightarrow$ a $\leftarrow$ 3

\emph{Iteration 2:} a = 3, b=3.5, accuracy = 0.5 \\
f(c) = f(3.25) $<$ 0 $\Rightarrow$ b $\leftarrow$ 3.25

\emph{Iteration 3:} a = 3, b=3.25, accuracy = 0.25\\
f(c) = f(3.125) $>$ 0 $\Rightarrow$ a $\leftarrow$ 3.125

\emph{Iteration 4:} a = 3.125, b=3.25, accuracy = 0.125\\
f(c) = f(3.1875) $<$ 0 $\Rightarrow$ b $\leftarrow$ 3.1875

\emph{Iteration 5:} a=3.125, b=3.1875, accuracy = 0.0625 $<$ 0.1.
$\Rightarrow$ $x^*=c=3.15625$

Grading:
\begin{itemize}
\item 1pt for method (no partial marks).
\item 1pt for numerical result (partial marks at each iteration. At iteration 5, any combination of a and b is accepted as solution).
\end{itemize}
}

\newpage
% 7. False position
\question{\underline{Using the false position (regula falsi) method}, find a zero of function $f(x)=\cos(x/2)$ assuming an initial bracket of [2.5, 3.5],
and using an accuracy of 0.01.}

\answer{
The false position method iterates as follows:
$$
c = \frac{af(b)-bf(a)}{f(b)-f(a)}
$$
This can be retrieved from the equation of the straight line passing by (a, f(a)) and (b, f(b)).

\emph{Iteration 1:}

$c_1 = \frac{2.5\cos(3.5/2)-3.5\cos(2.5/2)}{\cos(3.5/2)-\cos(2.5/2)} \approx 3.1389$.
f($c_1$) $>$ 0 so we replace a by $c_1$ in the next iteration.

\emph{Iteration 2:} a = 3.1389, b=3.5
$c_2 \approx 3.1416$.
$|c_1 - c_2| < 0.01$, the method has converged, our estimate is $c_2 \approx 3.1416$.

Grading:
\begin{itemize}
\item 1pt for method (partial marks if the formula ``looks good").
\item 1pt for numerical result (partial marks at each iteration).
\end{itemize}
}

\newpage
% 8. Newton-Raphson (1 equation)
\question{\underline{Using the Newton-Raphson method}, find a zero of function $f(x)=\cos(x/2)$ assuming an initial estimate of 3,
and using an accuracy of 0.01.}

\answer{We recall the Newton-Raphson iteration from the lectures:
$$
x_{i+1} = x_i - \frac{f(x)}{f'(x)}
$$
(Remember, f' is in the denominator, ``we have a problem when f'(x) is 
0".) The formula can be retrieved by approximating f by the straight 
line of slope f'(x) passing by (x, f(x)), and finding $x_{i+1}$ where 
this line crosses $y=0$.

Here we have: $f'(x) = -\frac{1}{2}sin(x/2)$.

\emph{Iteration 1:} $x_0$=3
$$
x_1 = 3 + 2\frac{cos(1.5)}{sin(1.5)} = 3 + \frac{2}{\tan(1.5)} \approx 3.1418
$$
Accuracy = 0.1418, we keep going.

\emph{Iteration 2:}
$$
x_2 = x_1 + \frac{2}{\tan(x_1/2)} \approx 3.141592
$$
Accuracy $<$ 0.01, method converged.

Grading:
\begin{itemize}
\item 1pt for method (partial marks if the formula ``looks good").
\item 1pt for numerical result (partial marks at each iteration, 0 if derivative is wrong.)
\end{itemize}
}

\newpage
\question{Fill the missing part of the following Python function, between
the start of the \texttt{for} loop and the \texttt{raise} statement:}

\begin{minted}{python}
def newton_raphson(f, diff, init_x, tol, max_iter=1000):
    '''
    Returns a zero of function f.
    diff is the derivative of the f.
    init_x is the initial estimate.
    tol is the tolerance (accuracy) of the solution.
    max_iter is the desired maximal number of iterations.
    '''

    x = init_x
    for i in range(max_iter):































    raise Exception("Unable to find a root")
\end{minted}

\answer{
As in the lecture notes.

Grading:
\begin{itemize}
\item 2/3pt for expression of delta (partial marks if it ``looks good")
\item 2/3pt for assignment of x (no partial mark)
\item 2/3pt for stop condition (no partial mark)
\end{itemize}
}

%%% Chapter 5

\newpage
% 10. Derivative (central)
\question{Given the data:\\
\begin{tabular}{|c|c|c|c|c|c|}
\hline
x    & 0.1 & 0.2 & 0.3 & 0.4 & 0.5\\
\hline
f(x) & 0.01 & 0.04 & 0.09 & 0.16 & 0.25\\
\hline
\end{tabular}\\
calculate f'(0.3) using an $\mathcal{O}(h^2)$ approximation.
}

\answer{
The points are evenly spaced, so we can use finite differences. Here, $h=0.1$.

We recall the expression of the central difference approximation of f':
$$
f'(x) = \frac{f(x+h)-f(x-h)}{2h} + \mathcal{O}(h^2)
$$
This expression is easy to remember as it is the slope of the straight line
passing by (x-h, f(x-h)) and (x+h, f(x+h)). It can also be retrieved by subtracting the
Taylor development of f in (x+h) and in (x-h).

We get:
$$
f'(0.3) = \frac{f(0.4)-f(0.2)}{0.2} = \frac{0.16 - 0.04}{0.2} = \frac{0.12}{0.2} = 0.6
$$
To check our result, we note from the data that f(x) = 2x, which gives 
f'(x)=2x (approximation in $\mathcal{O}(h^2)$ or even $\mathcal{O}(h)$ 
are exact in this case). 


Grading:
\begin{itemize}
\item 1pt for method (partial marks if the formula ``looks good").
\item 1pt for result (no partial marks)
\end{itemize}
}

\newpage
% 11. Derivative (forward)
\question{Given the data in the previous question, calculate f'(0.1) using an
$\mathcal{O}(h^2)$ approximation.
}

\answer{
We cannot use central differences since there is no data before 0.1. We will use
forward differences instead.

We recall that the formula for forward differences in $\mathcal{O}(h^2)$
is based on the Taylor development of f in (x+h) and in (x+2h):
$$
f(x+h) = f(x) + hf'(x) + \frac{h^2}{2}f''(x) + \ldots
$$
and
$$
f(x+2h) = f(x) + 2hf'(x) + \frac{4h^2}{2}f''(x) + \ldots
$$

To eliminate the term in $h^2$, we have to consider $f(x+2h)-4f(x+h)$:
$$
f(x+2h) - 4f(x+h) = f(x) - 4f(x) + 2hf'(x) - 4hf'(x) + \ldots
$$

It gives:
$$
f'(x) = \frac{-f(x+2h)+4f(x+h)-3f(x)}{2h} + \mathcal{O}(h^2)
$$
Note: to retrieve the formula, we don't bother with $\mathcal{O}$s, we just 
remember that this approximation is in $\mathcal{O}(h^2)$. 

The equation gives:
$$
f'(0.1) = \frac{-f(0.3)+4f(0.2)-3f(0.1)}{0.2} = \frac{-0.09 + 4*0.04 - 3*0.01}{0.2} = \frac{0.04}{0.2} = 0.2
$$

Grading:
\begin{itemize}
\item 1pt for method (partial marks if the formula ``looks good").
\item 1pt for result (no partial marks)
\end{itemize}
}


\end{document}
