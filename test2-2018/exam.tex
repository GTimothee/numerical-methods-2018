\documentclass{llncs}

\usepackage{amsmath} % for equation*
\usepackage{wasysym} % for \Box
\usepackage{tipa} % for |
\usepackage{color}
\usepackage{hyperref}
\usepackage{graphicx}
\usepackage{minted}
\definecolor{darkgreen}{rgb}{0,0.7,0}

% Fix link colors
\hypersetup{
    colorlinks = true,
    linkcolor=red,
    citecolor=red,
    urlcolor=blue,
    linktocpage % so that page numbers are clickable in toc
}

\pagestyle{plain}

\newcounter{ques}
\setcounter{ques}{1}

\newcommand{\todo}[1]{\color{blue}\textbf{TODO:} #1\color{black}}
\newcommand{\myspace}[0]{\vspace*{0.25cm}}

\renewcommand{\question}[1]{\paragraph{}\textbf{Q\theques} - #1\stepcounter{ques} }

\newcommand{\answer}[1]{\color{red}\textbf{Solution:}\\#1\color{black}}
\title{COMP 361/5611, Elementary Numerical Methods \\ Test 2 \\ Fall 2018 \\ Friday, November 30, 2018 \\ Duration: 150 minutes}

%\def\arraystretch{2}

\author{Tristan Glatard\\
  \href{mailto:tristan.glatard@concordia.ca}{tristan.glatard@concordia.ca}\\
  \vspace*{0.3cm}
  }

\institute{Concordia University\\
  Department of Computer Science and Software Engineering}

\begin{document}

\maketitle

\section*{Instructions}
\begin{itemize}
\item All questions will receive equal points.
\item Answer all questions on these sheets in the space provided.
\item No books, notes or extra paper (except draft paper received from the instructor).
\item No cell phones, laptops or any other electronic devices except ENCS calculators.
\item This exam is 15 pages long, including the cover page. It has
  12 questions labeled from \textbf{Q1} to \textbf{Q12}. Check that your copy
  is complete.
%~ \item Grading scheme:
%~ \begin{itemize}
%~ \item Good method, good result: 4pts
%~ \item Good method, wrong result (e.g., calculation error): 3pts
%~ \item Wrong method, good result: 2pts
%~ \item Wrong method, wrong result: 0pts
%~ \end{itemize}
\end{itemize}

\myspace

\myspace

\hrulefill\\

\myspace

Consistent with the university regulations concerning cheating and plagiarism I will not cheat during this examination:

\myspace

\myspace

Student ID: \dotfill

\myspace

\myspace

First Name / Last Name: \dotfill

\myspace

\myspace

Signature: \dotfill

\myspace

\myspace

\hrulefill

\newpage
%%% CHAPTER 2
\question{Compute the inverse of the following matrix using Gauss elimination:
$$
\textbf{A}=
\begin{bmatrix}
5 & 1 & -2 \\
10 & 3 & -4 \\
-5 & 2 & 3
\end{bmatrix}
$$
}
\answer{
The inverse of \textbf{A} is obtained by solving the following system of equations:
$$
\textbf{A}\textbf{X}=\textbf{I}
$$
Using the augmented form:
$$
\left[
\begin{array}{ccc|ccc}
5  & 1 & -2 & 1 & 0 & 0 \\
10 & 3 & -4 & 0 & 1 & 0 \\
-5 & 2 &  3 & 0 & 0 & 1
\end{array}
\right]
$$
(2) $\leftarrow$ (2) - 2 $\times$ (1)
(3) $\leftarrow$ (3) + (1):
$$
\left[
\begin{array}{ccc|ccc}
5  & 1 & -2 & 1  & 0 & 0 \\
0  & 1 & 0  & -2 & 1 & 0 \\
0  & 3 &  1 & 1 & 0 & 1
\end{array}
\right]
$$
(3) $\leftarrow$ (3) - 3 $\times$ (1):
$$
\left[
\begin{array}{ccc|ccc}
5  & 1 & -2 & 1  & 0 & 0 \\
0  & 1 & 0  & -2 & 1 & 0 \\
0  & 0 &  1 & 7 & -3 & 1
\end{array}
\right]
$$
Finally, by back-substitution, we obtain:
$$
\textbf{$A^{-1}$}= 
\begin{bmatrix}
17/5 & -7/5 & 2/5 \\
-2 & 1 & 0 \\
7 & -3 & 1
\end{bmatrix}
$$
Grading:
\begin{itemize}
\item 1pt for method, partial marks if "left side" is fine but right side isn't modified correctly.
\item 1pt for numerical result, partial marks if some iterations are complete.
\end{itemize}
General comments on grading:
\begin{itemize}
\item Partial marks mean 0.5, we don't go to 0.25 in any question, unless otherwise mentioned.
\item Final mark /24 is converted /100 and rounded up to the nearest integer.
\end{itemize}
}

\newpage
(This page left intentionally blank)

\newpage

\question{Use Dolittle's decomposition to find \textbf{L} and \textbf{U} such that: 
$$
\textbf{A}=\textbf{LU} =
\begin{bmatrix}
2 & 1 & 4 \\
4 & 3 & 0 \\
2 & 1 & 7
\end{bmatrix}
$$
}

\answer{
We perform Gauss elimination and store the multipliers in place of the lower elements of the matrix.

\noindent (2) $\leftarrow$ (2) - 2 $\times$ (1) \\
(3) $\leftarrow$ (3) - (1)

[\textbf{L}$\backslash$ \textbf{U}] = 
$
\begin{bmatrix}
2 & 1 & 4 \\
2 & 1 & - 8 \\
1 & 0 & 3
\end{bmatrix}
$

which gives:
$$
\textbf{L} = \begin{bmatrix}
1 & 0 & 0\\
2 & 1 & 0 \\
1 & 0 & 1
\end{bmatrix}
\quad 
\textbf{U} = \begin{bmatrix}
2 & 1 & 4 \\
0 & 1 & -8 \\
0 & 0 & 3
\end{bmatrix}
$$
}

\newpage

%% CHAPTER 3
\question{We are looking for a natural cubic spline $s$ to interpolate \emph{three} 2D points $(x_0, y_0)$, $(x_1 ,y_1)$ and $(x_2, y_2)$.
\begin{enumerate}
\item Write the general expression of the natural cubic spline. How many unknowns are there in this expression?
\item Write the equations that will be used to find these unknowns, as a function of $x_i, y_i$. How many equations are there?
\end{enumerate}
}

\answer{
1. The natural cubic spline is defined by piece-wise cubic polynomials on $[x_0, x_1]$ and $[x_1, x_2]$:
$$
s(x)=\begin{cases}
s_1(x) = a + bx + cx^2 + dx^3 \quad \mathrm{if} \quad x \in [x_0, x_1] \\
s_2(x) = e + fx + gx^2 + hx^3 \quad \mathrm{if} \quad x \in [x_1, x_2]
\end{cases}
$$
There are 8 unknowns which are the coefficients of the two cubic polynomials.
2. The following 8 equations can be used to determine the 8 unknowns:
\begin{itemize}
\item Cubic spline is \emph{natural}: $s_1''(x_0) = 0$ and $s_2''(x_2)=0$ (2 equations).
\item Cubic spline \emph{interpolates the data points} and is \emph{continuous}: $s_1(x_0)=y_0$, $s_1(x_1)=y_1$, $s_2(x_1)=y_1$ and $s_2(x_2)=y_2$ (4 equations).
\item Second derivative is continuous: $s_1''(x_1)=s_2''(x_1)$ (1 equation).
\item First derivative is continuous: $s_1'(x_1)=s_2'(x_1)$ (1 equation).
\end{itemize}
Grading:
\begin{itemize}
\item 1pt for sub-question 1., no partial marks.
\item 1pt for sub-question 2, 1/8 for each equation.
\end{itemize}
}

\newpage
%% CHAPTER 4
\question{\emph{Using the Newton-Raphson method}, find the positive zero of function $f(x)=x^2-17$ assuming an initial estimate
of 4 and using an accuracy of 0.001.}
\answer{
The Newton-Raphson iteration is defined as follows:
$$
x_{i+1} = x_i - \frac{f(x_{i})}{f'(x_{i})}
$$
Here, $f'(x)=2x$, thus:
\begin{itemize}
\item $x_1$ = $x_0 - \frac{x_0^2-17}{2x_0}$ = 4.125 (accuracy = 0.125)
\item $x_2$ = $x_1 - \frac{x_1^2-17}{2x_1}$ = 4.123 (accuracy = 0.002)
\item $x_3$ = $x_1 - \frac{x_1^2-17}{2x_1}$ = 4.123 (accuracy $\leq$ 0.001)
\end{itemize}
Grading:
\begin{itemize}
\item 1pt for method (Newton-Raphson formula), no partial marks.
\item 1pt for numerical result, partial marks if some iterations are complete.
\end{itemize}
}

\newpage
%% CHAPTER 5
\question{Using the central difference approximation in 
$\mathcal{O}(h^2)$, evaluate the derivative of f(x)=cos(x) at $x=0.1$ with 4 decimal places, 
for 3 different values of $h$: 0.1, 0.2 and 0.3. Evaluate the error of your 
approximation with respect to the exact value of f'(x).}

\answer{The central difference approximation of f'(x) in $\mathcal{O}(h^2)$ is given by:
$$
f'(x) = \frac{f(x+h)-f(x-h)}{2h}
$$
\begin{itemize}
\item h=0.1: f'(0.1) $\approx \frac{cos(0.2)-cos(0)}{0.2} \approx -0.0997   $
\item h=0.2: f'(0.1) $\approx \frac{cos(0.3)-cos(-0.1)}{0.4} \approx -0.0992$
\item h=0.3: f'(0.1) $\approx \frac{cos(0.4)-cos(-0.2)}{0.6} \approx -0.0983$
\end{itemize}

\textbf{Error}. the exact value of the derivative is given by:
$$
f'(x) = -sin(x),
$$
which gives $f'(0.1)\approx-0.9983$:
The error is:
\begin{itemize}
\item h=0.1: $E \approx 0.0002$
\item h=0.2: $E \approx 0.0007$
\item h=0.3: $E \approx 0.0015$
\end{itemize}

Grading:
\begin{itemize}
\item 1pt for method (central differences formula), no partial marks.
\item 1pt for numerical result, partial marks if some values are correct.
\end{itemize}
}

\newpage
\question{Using the 2-node Gauss-Legendre method, evaluate the following integral and evaluate the error by comparison to the exact result:
$$
\int_0^1 e^{-2t}dt
$$
Note: the Gauss-Legendre table is given in Appendix~\ref{app:gl}.
}
\answer{First we do the following change of variable to transform the integral to [-1, 1]:
$$
t = \frac{1}{2} + \frac{1}{2}x \quad ; \quad
dt = \frac{1}{2}dx
$$
It gives:
$$
I = \frac{1}{2}\int_{-1}^1 e^{-1-x}dx
$$
We can now apply the 2-node Gauss-Legendre method with $f(x)=\frac{1}{2}e^{-1-x}$:
$$
 I \approx f\left(\frac{1}{\sqrt{3}}\right) + f\left(-\frac{1}{\sqrt{3}}\right) \\
   \approx \frac{1}{2}e^{-1-\frac{1}{\sqrt{3}}} + \frac{1}{2}e^{-1+\frac{1}{\sqrt{3}}}
   \approx 0.4309
$$

\textbf{Error.} The exact value of the integral is:
$$
\int_0^1 e^{-2t}dt = \left[ \frac{e^{-2t}}{-2}\right]_0^1 \\
= \frac{e^{-2}}{-2}+ \frac{1}{2} \approx 0.4323
$$
The error of the Gauss-Legendre approximation is 0.0014.

Grading:
\begin{itemize}
\item 1pt for method, partial marks for change of variable and use of table.
\item 1pt for numerical result, no partial marks.
\end{itemize}
}

\newpage
\question{Estimate the integral of the previous question using the composite trapezoidal rule with 10 panels (n=10). Estimate the error w.r.t the exact value
computed in the previous question.}

\answer{The composite trapezoidal rule is given by:
$$
I = \frac{h}{2}\left[f(x_0) + 2f(x_1) + 2f(x_2) + \ldots + 2f(x_{n-1}) + f(x_n)\right]
$$
Here n=10, $x_0$=0 and $x_n$=$x_{10}$=1. So $h=0.1$ and $x_i$=$0.1i$.
It gives:
$$
I = \frac{0.1}{2}\left( e^{0} + 2e^{-0.2} + 2e^{-0.4} + 2e^{-0.6} + 2 e^{-0.8} + 2e^{-1.0} + 2e^{-1.2} + 2e^{-1.4} + 2e^{-1.6} + 2e^{-1.8} + e^{-2} \right)
  \approx 0.4338
$$
Error is $|0.4338-0.4323|=0.0015$.

Grading:
\begin{itemize}
\item 1pt for method, partial marks if formula "looks good".
\item 1pt for numerical result, no partial marks.
\end{itemize}
}

\newpage
\question{Write the code of the following Python function that implements numerical integration using the composite trapezoidal rule with $n$ panels:}
\begin{minted}{python}
from numpy import arange
def trapezoid(f, a, b, n):
    '''
    Integrates f between a and b using n panels (n+1 points)
    '''
\end{minted}

\vspace*{2cm}
\answer{As in the lecture notes, function trapezoid(f, a, b, n) in Ch 6.
Grading:
\begin{itemize}
\item 0.5 for the expression of h
\item 0.5 for the xi values
\item 0.5 for the first and last terms in I
\item 0.5 for the other terms in I (for loop)
\end{itemize}
No partial marks.
}

\newpage
\question{Consider the following initial-value problem:
$$
y'+2y-1=2x \quad ; \quad y(0)=1
$$
\begin{enumerate}
\item Verify that $y(x) = e^{-2x} + x$ is solution of the problem.
\item Express the problem in the form $y'=f(y, x)$ so that it can be solved by a numerical solver.
\end{enumerate}
}

\answer{
1. If $y = e^{-2x} + x$ then $y' = -2e^{-2x} + 1$ and we have:
$$
y'+2y-1 = -2e^{-2x} + 1 + 2e^{-2x} +2x -1 = 2x
$$
In addition, $y(0) = e^0 + 0 = 1$.
So $y$ is solution of the problem.

2. The expression of $f$ is obtained directly from the problem formulation: $f(x, y)=2x-2y+1$.

Grading:
\begin{itemize}
\item 1pt for sub-question 1, partial marks for equation and initial condition.
\item 1pt for sub-question 2, no partial marks.
\end{itemize}
}
\newpage

\question{Write a Python program to solve the initial-value problem (IVP) defined in the previous question using the Euler method. Your program must:
\begin{itemize}
\item Implement a function to solve an IVP using the Euler method, starting from the skeleton below.
\item Use function \texttt{euler} to solve the problem.
\end{itemize}
}
\begin{minted}{python}
from numpy import array
def euler(F, x0, y0, x, h):
    '''
    Return y(x) given the following initial value problem:
    y' = F(x, y)
    y(x0) = y0 # initial conditions
    h is the increment of x used in integration
    F = [y'[0], y'[1], ..., y'[n-1]]
    y = [y[0], y[1], ..., y[n-1]]
    '''
    X = [] # will store the value of x0 at each iteration
    Y = [] # will store the value of y0 at each iteration

















    return array(X), array(Y)
\end{minted}

\answer{Function \texttt{euler} is as in the lecture notes:

It can be used as follows to solve the problem in the previous question:}
\begin{minted}{python}
def f(x, y):
    return 2*x+1-2*y  # as in the second part of the previous question
X, Y = euler(f, 0, 1, 10, 0.001)
\end{minted}

\answer{
\color{red} Grading:
\begin{itemize}
\item 1pt for \texttt{euler} function: 0.5pt for overall principle (while loop), 0.5pt for correct update of y using f.
\item 1pt for using the \texttt{euler} function: 0.5pt for function definition, 0.5pt for correct use with initial condition.
\end{itemize}
}
\newpage

\question{We consider the following two-point boundary problem:
$$
y''+2y'+y=1 \quad ; \quad y(0)=0  \quad \mathrm{and} \quad y(1)=1
$$
Solve the problem using the \emph{finite difference method} with 3 nodes (m=2). 
}

\answer{The unknowns of the problem are $y_0$, $y_1$ and $y_2$. We know that $y_0=0$ and $y_2=1$ from the boundary conditions. The finite difference approximation
of $y_1'$ and $y_1''$ give:
$$
y_1''=\frac{y_0-2y_1+y_2}{h^2} \quad \mathrm{and} \quad y_1' = \frac{y_2-y_0}{2h}
$$
With $h=1/2$.
Therefore the problem is expressed as follows:
$$
\frac{y_0-2y_1+y_2}{h^2} + 2\frac{y_2-y_0}{2h}+y_1=1
$$
which gives:
$$
y_1 = \frac{5}{7}
$$

Grading:
\begin{itemize}
\item 1pt for problem formulation, partial marks if principle is good but formulas are approximate.
\item 1pt for numerical result, no partial marks.
\end{itemize}
}

\newpage
\question{Explain the principle of the shooting method to solve 
two-point boundary problems, and write the main steps of the 
algorithm.}

\answer{
Grading:
\begin{itemize}
\item 2/3 pt for expression of residual function,
\item 2/3 pt for correct mentioning of the IVP problem
\item 2/3 pt for zero-finding of residual function (False position)
\end{itemize}
}


\newpage
\appendix

\section{Gauss-Legendre table}
\label{app:gl}
\begin{tabular}{c|c|c|c}
n & $x_i$                                                                                                          & $A_i$                                                & Error                             \\ 
\hline                                                                                                                                                                                                         
1 & $\pm\frac{1}{\sqrt{3}}$                                                                                        & 1                                                    & $\frac{f^{(4)}(\xi)}{135}$        \\
2 & 0 ; $\pm \sqrt{\frac{3}{5}}$                                                                                   & $\frac{8}{9}$ ; $\frac{5}{9}$                        & $\frac{f^{(6)}(\xi)}{15,750}$     \\
3 & \parbox{5cm}{$\pm \sqrt{\frac{3}{7}-\frac{2}{7}\sqrt{\frac{6}{5}}}$ \\  $\pm \sqrt{\frac{3}{7}+\frac{2}{7}\sqrt{\frac{6}{5}}}$} & \parbox{5cm}{$\frac{18+\sqrt{30}}{36}$ \\ $\frac{18-\sqrt{30}}{36}$} & $\frac{f^{(8)}(\xi)}{3,472,875}$  \\
\end{tabular}
\end{document}
